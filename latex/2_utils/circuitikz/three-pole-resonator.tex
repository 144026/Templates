% !TEX program = xelatex
\documentclass{article}

% geometry, color and font
\usepackage[tmargin=1in,bmargin=1in,lmargin=1in,rmargin=1in]{geometry}
\usepackage{xcolor}
\usepackage{fontspec}
\usepackage{amsmath}

% links
\usepackage{hyperref}
\hypersetup{
	colorlinks=true,
	linkcolor=black,
	filecolor=black,
	urlcolor=blue,
	citecolor=cyan,
}

% figures
\usepackage{graphicx}
\usepackage{tikz}

% tables

% code snippets(listings)
\usepackage{listings}
\lstset{
	showstringspaces=false,
	tabsize=4,
	%numbers=left,
	backgroundcolor=\color[RGB]{245,245,245},
	%
	basicstyle=\small\setmainfont{Fira Code Retina},
	keywordstyle=\color[RGB]{40,40,255},
	commentstyle=\color[RGB]{0,96,96},
	stringstyle=\color[RGB]{128,0,0},
}


% NOTE: '\usepackage{circuitikz}[#1,#2,...]' format NOT supported!!!
\usepackage[americaninductors,europeanresistors]{circuitikz}

\begin{document}

\setlength{\baselineskip}{32pt}

% 
\begin{circuitikz}
	\ctikzset{bipoles/length=.8cm,resistors/scale=0.8}
% 	\draw[color=gray!40] (-2,-2) grid (2,2);
	\draw (0,0) node[npn](t){};
	\draw (t.E) to[short] (0,-1) node[rground]{};
	\draw (t.C) to[short] (0,0.5) to[short] (1,0.5) 
	to[short] (1,-2) to[R,l^=$X_{be}$] (-1,-2) to[short] (-1,0) to (t.B);
	\draw (1,-1) to[R,l^=$X_{ce}$,*-*] (0,-1) to[R,l^=$X_{bc}$,*-*] (-1,-1);
\end{circuitikz}

\bigskip
% 三端电感
\begin{circuitikz}
	\ctikzset{bipoles/length=.8cm}
% 	\draw[color=gray!40] (-2,-2) grid (2,2);
	\draw (0,0) node[npn](t){};
	\draw (t.E) to[short] (0,-1) node[rground]{};
	\draw (t.C) to[short] (0,0.5) to[short] (1,0.5) 
		to[short] (1,-2) to[C] (-1,-2) to[short] (-1,0) to (t.B);
	\draw (1,-1) to[L,*-*] (0,-1) to[L,*-*] (-1,-1);
\end{circuitikz}

\bigskip
%三端电容
\begin{circuitikz}
	\ctikzset{bipoles/length=.8cm}
% 	\draw[color=gray!40] (-2,-2) grid (2,2);
	\draw (0,0) node[npn](t){};
	\draw (t.E) to[short] (0,-1) node[rground]{};
	\draw (t.C) to[short] (0,0.5) to[short] (1,0.5) 
		to[short] (1,-2) to[L] (-1,-2) to[short] (-1,0) to (t.B);
	\draw (1,-1) to[C,*-*] (0,-1) to[C,*-*] (-1,-1);
\end{circuitikz}

\bigskip
%串联改进型电容
\begin{circuitikz}
	\ctikzset{bipoles/length=.8cm}
% 	\draw[color=gray!40] (-2,-2) grid (2,2);
	\draw (0,0) node[npn](t){};
	\draw (t.E) to[short] (0,-1) node[rground]{};
	\draw (t.C) to[short] (0,0.5) to[short] (1,0.5) 
		to[short] (1,-2) to[C] (0,-2) to[L] (-1,-2) 
		to[short] (-1,0) to (t.B);
	\draw (1,-1) to[C,*-*] (0,-1) to[C,*-*] (-1,-1);
\end{circuitikz}

\bigskip
%并联改进型电容三端
\begin{circuitikz}
	\ctikzset{bipoles/length=.8cm}
% 	\draw[color=gray!40] (-2,-2) grid (2,2);
	\draw (0,0) node[npn](t){};
	\draw (t.E) to[short] (0,-1) node[rground]{};
	\draw (t.C) to[short] (0,0.5) to[short] (1,0.5) 
		to[short] (1,-2) to[C] (0,-2) 
		to[short] (0,-1.6) to[L] (-1,-1.6) 
		to[short] (-1,0) to (t.B);
		\draw (0,-2) to[short,*-] (0,-2.3) to[vC] (-1,-2.3) to[short] (-1,-2.3) to[short,-*] (-1,-1.6);
	\draw (1,-1) to[C,*-*] (0,-1) to[C,*-*] (-1,-1);
\end{circuitikz}

\bigskip

\end{document}
