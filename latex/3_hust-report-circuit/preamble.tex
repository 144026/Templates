%----------geometry, color and font------
\usepackage[tmargin=1in,bmargin=1in,lmargin=1in,rmargin=1in]{geometry}
\usepackage{xcolor}
\usepackage{ctex}
%\usepackage{fontspec} contained by ctex
\usepackage{amsmath}
%----------geometry, color and font------


%----------headers------------------------
\usepackage{fancyhdr}
\fancyhead[L]{\small \kaishu \LeftHeader}
\fancyhead[R]{\small \kaishu \RightHeader}
%----------headers------------------------


%----------figures-----------------------
\usepackage{graphicx}
\usepackage{tikz}
\usepackage[americaninductors,europeanresistors]{circuitikz}
%----------figures-----------------------


% tables


%----------captions----------------------
\usepackage{float}
\usepackage[
    font=small, %
    labelfont=bf, %
    textfont=normalfont, %
    labelformat=simple, %
    labelsep=period %
]{caption}
\usepackage[
    font=small, %
    labelfont=normalfont, %
    textfont=normalfont, %
    labelformat=parens, %
    labelsep=space %
]{subcaption}
\renewcommand{\figurename}{图}
\renewcommand{\tablename}{表}
%----------captions----------------------


%----------xref--------------------------
\usepackage{hyperref}
\hypersetup{
    colorlinks=true,
    linkcolor=black,
    filecolor=black,
    urlcolor=blue,
    citecolor=cyan
}
\renewcommand{\figureautorefname}{图\hspace{-3pt}}
\renewcommand{\subfigureautorefname}{图\hspace{-3pt}}
\renewcommand{\tableautorefname}{表\hspace{-3pt}}
\renewcommand{\subtableautorefname}{表\hspace{-3pt}}
%----------xref--------------------------


%----------toc------------------------------ 
\usepackage[nottoc]{tocbibind} % add ref to toc
\setcounter{tocdepth}{2}
\usepackage{setspace}

\usepackage{tocloft}
\renewcommand{\contentsname}{目\ 录}
\tocloftpagestyle{plain} % this package overwrites pagestyle!
%----------toc------------------------------ 



%----------appendix------------------------
\usepackage[page, toc, title, titletoc]{appendix} % page(appendixpagename), toc(appendixtocname)
\renewcommand{\appendixname}{附录}
\renewcommand{\appendixtocname}{附录}
\renewcommand{\appendixpagename}{附录}
%----------appendix------------------------


%----------code snippets(listings)---------

\usepackage{listings}
\lstset{
    showtabs=false,
    showspaces=false,
    showstringspaces=false,
    numbers=left,
    tabsize=4,
    keepspaces=true,
    breaklines=true,
    backgroundcolor=\color[RGB]{245,245,245},
    basicstyle=\small\setmainfont{Courier New},
    keywordstyle=\color[RGB]{40,40,255},
    commentstyle=\color[RGB]{0,96,96},
    numberstyle=\scriptsize\texttt,
    stringstyle=\color[RGB]{128,0,0}
}

%----------code snippets(listings)---------


%----------chinese section-----------------
\usepackage{titlesec}
\titleformat*{\section}{\fontsize{16pt}{\baselineskip}\bfseries\songti}
\titleformat*{\subsection}{\fontsize{14pt}{\baselineskip}\bfseries\songti}
\titleformat*{\subsubsection}{\fontsize{12pt}{\baselineskip}\bfseries\songti}
%----------chinese section-----------------


%----------marcos--------------------------
\newcommand{\LeftHeader}{通信电子线路}
\newcommand{\RightHeader}{xxxx实验报告}
%----------marcos--------------------------
