% !TEX program = xelatex
\documentclass{beamer}
\mode<presentation>
\usetheme{Boadilla}

\title{this is a title}
\author{author}
\institute{institute}
\date{\today}

\usepackage{listings}
\lstset{ %
  backgroundcolor=\color{white},
  basicstyle=\tiny\ttfamily,
  breakatwhitespace=false,
  breaklines=true,
  captionpos=b,
  commentstyle=\color{green},
  escapeinside={\%*}{*)},
  extendedchars=true,
  frame=single,
  keywordstyle=\color{blue},
  numbers=left,
  numbersep=5pt,
  numberstyle=\tiny\color{gray},
  rulecolor=\color{black},
  showspaces=false,
  showstringspaces=false,
  showtabs=false,
  stepnumber=1,
  stringstyle=\color{mauve},
  tabsize=2,
  title=\lstname,
}

\begin{document}

\begin{frame}
\titlepage
\end{frame}


% \defverbatim[colored]\mycode{
% \begin{lstlisting}[language=C++]
% void make_set(int X) {
%   parent[X] = X;
% }
% \end{lstlisting}
% }

\begin{frame}
\frametitle{frame one}

\begin{itemize}
\item
point one
\item
point two
\item
point three

point three line two
\item
point four
\end{itemize}

\end{frame}

\begin{frame}[fragile]
\frametitle{VEC or vector?}

\begin{lstlisting}[language=C++]
/* C */
typedef struct loop *loop_p;
DEF_VEC_P (loop_p);
DEF_VEC_ALLOC_P (loop_p, gc);

  VEC (loop_p, gc) *superloops;
  VEC_reserve (loop_p, gc, superloops, depth);
  VEC_index (loop_p, superloops, depth)
  VEC_quick_push (loop_p, superloops, father);
\end{lstlisting}

\begin{uncoverenv}<2->
\begin{lstlisting}[language=C++]
// C++
typedef std::vector<struct loop*, gc_allocator> loop_vec;
  loop_vec* superloops;
  superloops->reserve(depth);
  superloops[depth];
  superloops->push_back(father);
\end{lstlisting}
\end{uncoverenv}

\end{frame}


\begin{frame}
\frametitle{Why not C++?}

\begin{itemize}
\item
{\color{red} C++ is too slow!}
\begin{uncoverenv}<2->
\begin{itemize}
{\color{green!50!black}
\item
C++ is only slower when using optional features which aren't in C.
\item
Sometimes C++ is faster (e.g., STL functions).
\item
We would only use features which are worthwhile.
}
\end{itemize}
\end{uncoverenv}
\item
{\color{red} C++ is too complicated!}
\begin{uncoverenv}<3->
\begin{itemize}
{\color{green!50!black}
\item
It's just another computer language.
\item
Maintainers will ensure that gcc continues to be maintainable.
}
\end{itemize}
\end{uncoverenv}
\item
{\color{red} C++ library is a bootstrap problem!}
\begin{uncoverenv}<4->
\begin{itemize}
{\color{green!50!black}
\item
C++ compilers are widely available, including older versions of gcc.
\item
We would have to ensure that gcc version N - 1 could always build gcc
version N.
\item
We will link statically against {\tt libstdc++}.
}
\end{itemize}
\end{uncoverenv}
\item
{\color{red} The FSF doesn't like it!}
\begin{uncoverenv}<5->
\begin{itemize}
{\color{green!50!black}
\item
The FSF is not writing the code.
}
\end{itemize}
\end{uncoverenv}

\end{itemize}

\end{frame}

\begin{frame}
\frametitle{Conclusion}

\begin{itemize}
\item
point one
\item
point two
\begin{itemize}
\item
subpoint one
\end{itemize}
\end{itemize}


\begin{block}{a block}
this is a block
\end{block}

\begin{alertblock}{an alert}
this is an alert
\end{alertblock}

\begin{exampleblock}{an example}
this is an example
\end{exampleblock}

\end{frame}

\end{document}
