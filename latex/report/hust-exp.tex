% !TEX program = xelatex
\documentclass[12pt]{article}

\input{/home/drh/Templates/latex/preamble-core-ch.tex}

\usepackage[americaninductors,europeanresistors]{circuitikz}

\begin{document}
\pagestyle{empty}
\

{
\vspace{70pt}
\centering\fontsize{24pt}{\baselineskip}\textbf{xxxx仿真}

\vspace{35pt}
\centering\fontsize{32pt}{\baselineskip}\textbf{实验报告}

}

\vspace{155pt}
\begin{table}[htbp]
	\centering
	\begin{tabular}{ll}
        学院:\quad  &\underline{\qquad 电子信息与通信学院\hspace{0.5cm} } \bigskip \\
		班级:\quad  &\underline{\qquad 种子1801班\hspace{1.88cm} } \bigskip \\
        时间:\quad  &\underline{\qquad 2020年11月16日\hspace{0.642cm} } \bigskip \\
        学号:\quad  &\underline{\qquad U201813319\hspace{2.044cm} } \bigskip \\
        姓名:\quad  &\underline{\qquad 董瑞华\hspace{2.975cm} } \bigskip \\
% 		姓名:\quad  &\underline{\qquad \hspace{36pt}\hspace{2.975cm} } \bigskip \\
	\end{tabular}
\end{table}
\newpage
\pagestyle{plain}



\section{实验目的}

\begin{enumerate}
	\item aaa
	\item bbb
	\item ccc
\end{enumerate}


\section{实验原理和设计}

\subsection{实验原理}

电路如图\ref{fig:cprin}所示。

\begin{figure}[htbp]
	\centering
	\begin{circuitikz}[scale=1.5]
		\small
		\ctikzset{
			bipoles/length=1.2cm,
			resistors/scale=0.7,
			inductors/scale=0.7,
			capacitors/scale=0.8,
			monopoles/vcc/arrow={|}
		}
		
		%       \draw[color=gray!40] (-5,-5) grid (5,5);
		\draw (0,0) node[npn](t){Q1};
		\draw (t.E) to[short] (0,-0.5) to[R,l_=$R_E$] (0,-2) node(gnd)[rground]{}
			(0,-0.5) to[short,*-] (1,-0.5) to[C,l_=$C_E$] (1,-2) to[short,-*] (0,-2);
		
		\draw (t.C) to[short] (0,0.5) to[C,l=$C$] (0,2) node(vcc)[vcc]{$V_{CC}$}
			(0,0.5) to[short,*-] (1,0.5) to[R,l=$R$] (1,2) to[short,-*] (0,2)
			(1,0.5) to[short,*-] (2,0.5) to[L,l=$L$] (2,2) to[short,-*] (1,2)
			(2,0.5) to[C,l=$C_{out}$,*-o] (3,0.5) node[right]{$V_{out}$};
		
		\draw (t.B) to[short] (-0.5,0) to[short,-*] (-1,0)
			to[C,l=$C_{in}$,-o] (-2,0) node[left]{$V_{in}$}
			(-1,0) to[R,l=$R_{b1}$] (-1,2) to[short,-*] (0,2)
			(-1,0) to[R,l_=$R_{b2}$] (-1,-2) to[short,-*] (0,-2);
	\end{circuitikz}
	\caption{高频谐振小信号放大电路}
	\label{fig:cprin}
\end{figure}

\subsection{实验设计}

Multisim仿真电路如图\ref{fig:csim}所示。

\begin{figure}[htbp]
	\centering
	\includegraphics[scale=0.3]{figures/144026.png}
	\caption{Multisim仿真电路}
	\label{fig:csim}
\end{figure}


\section{实验步骤}

\begin{enumerate}
	\item 连线
	\item 运行直流偏置点仿真,结果如图\ref{fig:simdc}所示。仿真结果为$I_c \approx I_e \approx 100I_b$。本实验中设置$Q1$的放大系数$\beta=100$,因此结果符合预期,$Q1$工作在放大区。

\begin{figure}[htbp]
	\centering
	\includegraphics[scale=0.3]{figures/144026.png}
	\caption{直流偏置点}
	\label{fig:simdc}
\end{figure}

	\item 连线
	\item 连线
	\item 连线
\end{enumerate}
	

\section{实验结果及分析}

\section{实验小结}

\end{document}
